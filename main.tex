% This is samplepaper.tex, a sample chapter demonstrating the
% LLNCS macro package for Springer Computer Science proceedings;
% Version 2.20 of 2017/10/04
%
\documentclass[runningheads]{llncs}
%
\usepackage{graphicx}
%\usepackage{optidef}
\usepackage{amsmath}
%\usepackage[numbers]{natbib}


% Used for displaying a sample figure. If possible, figure files should
% be included in EPS format.
%
% If you use the hyperref package, please uncomment the following line
% to display URLs in blue roman font according to Springer's eBook style:
% \renewcommand\UrlFont{\color{blue}\rmfamily}

\newcommand{\li}[1][]{x^i_{#1}}
\newcommand{\oi}[1][]{n^i_{#1}}
\newcommand{\bsi}[1][]{b^i_{#1}}
\newcommand{\bai}[1][]{s^i_{#1}}
\newcommand{\ci}[1][]{C^i_{#1}}

%\newcommand{\pbi}[1][]{p^{b(i)}_{#1}}
\newcommand{\pbi}[1][]{\beta^i_{#1}}
\newcommand{\pzi}[1][]{\zeta^i_{#1}}
%\newcommand{\pzi}[1][]{p^{s(i)}_{#1}}

\newcommand{\fsi}{\mathcal{F}^i}
\newcommand{\storagei}{\mathbf{S}^i}
\newcommand{\bini}{\mathbf{S}^i_0}
\newcommand{\guari}{\alpha^i}

\newcommand{\mmp}[1]{[#1]^+}
\newcommand{\mmm}[1]{[#1]^-}

\newcommand{\pri}[1][]{\lambda^i_{#1}}
\newcommand{\zi}[1][]{z^i_{#1}}

\newcommand{\gaini}[1][]{\mathcal{P}^i_{#1}}
\newcommand{\resi}{\alpha^i}

\newcommand{\N}{\mathcal{N}}
\newcommand{\T}{\mathcal{T}}

\newcommand{\uti}{u^i}

%%\newcommand{\}{}


\begin{document}
%
\title{Design of a combinatorial double auction for local energy markets}
%
%\titlerunning{Abbreviated paper title}
% If the paper title is too long for the running head, you can set
% an abbreviated paper title here
%
\author{Diego Kiedanski \inst{1} \and
Daniel Kofman\inst{1} \and
Ariel Orda \inst{2}
}
%
\authorrunning{Kiedanski et al.}
% First names are abbreviated in the running head.
% If there are more than two authors, 'et al.' is used.
%
\institute{Télécom ParisTech \and
Technion }
%
\maketitle              % typeset the header of the contribution
%
\begin{abstract}

Local energy markets allow neighbours to exchange energy among them. Their traditional implementation using sequential auctions has proven to be inefficient and even counterproductive in some cases. In this paper we propose a combinatorial double auction for the exchange of energy for several time-slots simultaneously.
We suppose that participants have a flexible demand; flexibility being obtained, for example, by the usage of a battery. We show the benefits of the approach and we provide an example of how it can improve the utility of all the participants in the market. 

\keywords{Auction, Smart Grids, Local Energy Markets}
\end{abstract}
%
%
%

\section{Introduction}

Local energy markets (LEMs) have been proposed as a paradigm to better exploit the benefits of distributed local energy generation. The various proposed market mechanisms target to encourage  neighbours to exchange energy locally - within the same low voltage distribution grid, for example - in order to reduce their energy bill or even to generate revenue. In most implementations, the market mechanisms consist of a sequence of auctions that allow the participants to trade energy for the next time-slot (usually 15 or 30 minutes long). For a review of different proposals and implementations, the reader is referred to \cite{weinhardt2019far}, \cite{lopez2016methods} and the references therein. 
LEMs are usually implemented as double auctions, with players (households) submitting both buying and selling bids. In particular, a house with renewable generation can be a buyer or a seller, depending on the time-slot. 
In addition, if households have flexibility in their consumption profiles (for example, thanks to energy storage systems), they will schedule their load to obtain the most out of the market. 
In spite of this, it is known that the system architecture involving sequential auctions does not fully exploit the available flexibility and can even be counterproductive in some cases \cite{kok2005powermatcher}. 
In this paper, we put forward the design of an approach based on a combinatorial double auction \cite{LI200959}, \cite{SAMIMI2016201}, \cite{XIA2005239} that improves the utility of all players and increases the total traded energy.
%therefore reducing the exchanges of energy outside the borders of the local distribution grid (reducing flows for which the traditional energy company (TEC) is involved). 


% Local energy markets (LEMs) have been proposed to exploit the benefits of distributed energy generation. This are market mechanisms that enable neighbours to exchange energy locally (within the same low voltage distribution grid, for example). In most implementation, they consists on a sequence of auctions that allow the participants to trade energy for the next time-slot (usually 15 or 30 minutes long). For a review of different implementations and proposal, the reader is refered to \cite{weinhardt2019far} and the references therein.

% Unlike wholesale electricity markets where the demand is assumed to be fixed, LEMs are usually implemented as double auctions, with players (households) submitting buying and selling bids. In particular, a house with renewable generation can be both, a buyer and a seller in different time-slots.

% If households have flexibility in their consumption profiles (for example, thanks to energy storage systems), they will schedule their load to obtain the most out of the market and minimize their costs. In spite of this, it is known that the system architecture involving sequential auctions does not fully exploit the available flexibility.

% In this paper, we put forward the design of a combinatorial auction that improves the utility of all players will increasing the traded energy in the grid.

% With more renewables we want more LEMs. In LEMs, end-consumers can exchange energy among them. Traditionally, a sequence of auctions is used to implement LEMs. Unfortunatelly, it has been shown that such implementation is inefficient and fails to properly syncronize the participants.
% Motivated by the combinatorial bids in the wholesale electricity markets, we present a combinatorial auction for trading energy during a whole day, aimed at replacing the sequence of smaller auctions.
% To model the flexibility in demand of the different participants, we resort to the use of energy storage.

% Our proposal consists on replacing T sequential double auctions with one combinatorial auction to be run at the beginning of the day. 


\section{Mathematical model of players}

Let $\N$ denote the number of players and $\T$ the number of time-slots in a given day. Each player can consume energy buy using appliances (water heater, A/C, charging electric vehicules, TV, etc.) and might produce energy (e.g. photovoltaic generation). Let $\li[t]$ denote the demand of player $i$ at time-slot $t$, where a positive value of $\li[t]$ represents excess of consumption while a negative value stands for a surplus of renewable energy (the definition of $x$ is independent of possible flows with a battery, those flows will be introduced through additional variables). The demand profile $\li=(\li[1], \dots, \li[T])$ of player $i$ is assumed fixed and known. 

To simplify the presentation, we suppose that the flexibility of each player is introduced only by batteries (for example, the demand of the appliances is not shifted in time). Let $\storagei$ denote the total capacity of player i’s battery (possibly 0), $\bini$ the initial state of charge and $\bai[t]$ the amount of charged ($\bai[t] \geq 0$) or discharged ($\bai[t] < 0$) energy at time-slot $t$. 
The feasible set of charging/discharging decisions $\fsi$ is given by Equation \eqref{eq:feasible}. 


%Let $\N$ denote the number of players and $\T$ the number of time-slots in a given day. Each player can consume energy buy using appliances (such as TVs, water heaters, etc. or might produce energy (if she owns a photovoltaic-panel). Let $\li[t]$ denote the consumption level of player $i$ at time-slot $t$, where a positive value of $\li[t]$ represents excess of consumption while a negative value stands for a surplus or renewable energy. The load profile $\li$ of player $i$ is assumed fixed and known.
%We model the flexibility of each player using batteries. Let $\storagei$ denote the total capacity of player $i$'s battery (possibly 0), $\bini$ the initial state of charge and $\bai[t]$ the amount of charged ($\bai[t] \geq 0$) or discharged ($\bai[t] < 0$) energy at time-slot $t$.

%In a LEM, players are households that opt to buy and sell energy to neighbours in addition to trading with their traditional electricity company (TEC). Lets consider a day divided in $T$ time-slots. During time-slot $t$, player can $i$ consumes $\li[t]$ kWh (if $\li[t] \leq 0$) or have a surplus or renewable energy (if $\li[t] < 0$). Because players have storage, they can change their consumption profile by charging and discharging their battery. 
%We proceed to describe the model of a player with a battery representing and end-customer in a low voltage electricity grid, such as a household.

%Lets consider a battery that can be charged or discharged. 
%Let $\bsi[t]$ denote the battery state of charge and $\bai[t]$ the amount of energy charged (positive) or discharged (negative) in the battery. 
%The feasible set of charging/discharging decisions $\fsi$ is given by Equation \eqref{eq:feasible}. Furthermore, we will denote by $\oi = \bai + \li, \bai \in \fsi$ the net consumption of player $i$ as seen from the grid.
%Furthermore, let $\li[t]$ denote the energy consumed (positive) or in excess (negative) at time-slot $t$. The load profile is inflexible and all the change in consumption is driven by the battery. The feasible action space of the battery is given by:

\begin{equation}
    \label{eq:feasible}
\fsi = \left\{ \bai \colon \bini + \sum^{j}_{t=1} \bai[t] \in [0, \storagei], \ \forall j=1, \dots, T;  \bai \in \mathbb{R}^T  \right\}
\end{equation}

%\begin{equation}\label{eq:feasible}
%\fsi = \begin{cases}
%    \bsi[t + 1] = \bsi[t] + \bai[t] & t=1, \dots T\\
%    \bsi[t] \in [0, \storagei]  & t=1, \dots T \\
%    \oi[t] = \li[t] + \bai[t]  & t=1, \dots T \\
%\end{cases}
%\end{equation}

%where the decisions are exactly how much to charge or discharge the battery at every time-slot $\bai[t]$.
%In that case, the cost paid by player $i$ at time-slot $t$ is given by:
Furthermore, we will denote by $\oi = \bai + \li$, with $\bai \in \fsi$, the net consumption of player $i$ as seen from the grid. 

In addition to trading in the market, households can trade with their traditional electricity company (TEC). If we denote $\pbi[t]$ player i's cost of buying electricity from the TEC and $\pzi[t]$ the cost of selling to the TEC at time-slot $t$, then the cost at time-slot $t$ associated with a load of $w^i_t$ is given by:  $\ci[t](w_t) = \pbi[t]\max\{w^i_t, 0\} - \pzi[t]\max\{- w^i_t, 0\}$.
%Without the market, the cost of a player $i$ during time-slot $t$ is determined by her profile $\oi$ and the prices of the TEC as described in Equation \eqref{eq:cost}

%\begin{equation}\label{eq:cost}
%    \ci[t](w_t) = \pbi[t]\max\{w^i_t, 0\} - \pzi[t]\max\{- w^i_t, 0\}
%\end{equation}

%where $\pbi[t]$ is the cost of buying energy at time-slot $t$, $\pzi[t]$ is the cost of selling energy at time-slot $t$,
%where $\mmp{x} = \max\{x, 0\}$ and $\mmm{x} = \max\{-x, 0\}$. 
%\footnote{
%The above cost can be rewritten using a binary variable as follows:

%\begin{equation}
%    \begin{cases}
%    \ci[t](\oi[t]) = \pbi[t]w^i_t\zi[t] + \pzi[t]w^i_t(1 - \zi[t]) \\
%    w^i_t\zi[t] \geq 0 \\
    % w^i_t(1 - \zi[t]) \leq 0 \\
    % \zi[t] \in \{0, 1\}
    % \end{cases}
%\end{equation}}

\subsection{Utility of players trading in the market and with the TEC}

We introduce here the definition we use of the utility of any given player when trading in the local market. 

At time-slot $t$, player $i$ might be able to trade a fraction $\pri[t] \in [0, 1]$ of her net load $\oi[t]$ in the local market. If player $i$ trades $\pri[t]\oi[t]$ in the local market, then it will have to trade the quantity $(1-\pri[t])\oi[t]$ with the TEC. Denoting $\gaini$ the payment of player $i$ associated with the total quantity traded in the local market (positive if buying, negative if selling), the utility of player $i$ is given by: 


%So far, we described the set of feasible consumption profiles of players, and their associated cost. We will now explain the utility of a given player while trading in the market.

%Let $\oi$ be one of the possible net consumption profiles of player $i$. At time-slot $t$, player $i$ might be able to trade a fraction $\pri[t] \in [0, 1]$ of his load $\oi[t]$ in the market. If player $i$ trades $\pri[t]\oi[t]$ in the market, then it will have to settle the quantity $(1 - \pri[t])\oi[t]$ with the TEC. Denoting $\gaini[t]$ the payment associated with the traded quantity (positive if selling, negative if buying) of player $i$ at time-slot $t$, the utility of player $i$ is given by:

\begin{equation}
    \label{eq:utility}
\uti(\oi, \pri, \gaini) =  \begin{cases}

    -\gaini - \sum^{T}_{t=1}  \ci[t]((1 - \pri[t])\oi[t])  & \text{ if } (\oi - \li \in \fsi) \\
    - \infty & \text{ otherwise}

    \end{cases}  
\end{equation}

%Finally, lets define by $\guari$ as the maximum utility that player $i$ can guarantee without trading in the local market, i.e., $\displaystyle \guari = \max_{\bai \in \fsi} \uti(\bai + \li, 0, 0)$

%In our model, players seek to minimize their electricity bill, so that there utility of player $i$ is given by $\uti(\oi) = -\sum_{t} C^i_t(\oi[t])$.

%We shall assume that players have quasi-linear utilities, so that while trading in the market, their utility is given instead by $\uti(\oi) = -\sum_{t} \left[C^i_t(\oi[t]) -\gaini[t]\right]$, where $\gaini[t]$ is the payment associated with trades during time-slot $t$.


% \begin{example}
% Consider a player with load $\li = (0, 0, 0.9)$, $\pbi = (5, 4.9, 5)$, $\pzi=(1, 1, 1)$ and $\storagei = 1$. The charging/discharging decision that maximizes her utility if no trade occurs is given by $\bai = (0, 0.9, -0.9)$, for a net load of $\oi = (0, 0.9, 0)$ and a total utility of $\guari = -4.4$.
% \end{example}

%If we were to implement LEMs as sequential auctions, each player would first select her optimal strategy and would later attempt to trade. For the players in the example introduced above, it is clear that no trade would happen.

% The utility of player $i$ is quasi-linear on the cost paid and any payments he/she might receive to consume that specific profile and is given in equation \eqref{eq:utility}.

% \begin{equation}\label{eq:utility}
%     \uti \colon \fsi \to \mathbb{R}, \ \quad \uti(\bai) = -\left(\sum_{t=1}^T \ci[t](\bai[t]) - \gaini[t] \right)
% \end{equation}

% When players do not participate in the market, they do not receive payments ($\gaini[t] = 0$) and their utility is exactly the negative of their cost. 

\section{Auction model}

We put forward the design of a combinatorial double auction that exploits the flexibility available for players. Unlike the traditional auctions used for LEMs in which players bid the quantity they want to buy or sell for a single time-slot, we allow players express in their bids they desire to acquire specific profiles of energy spanning multiple periods. We proceed to explain the bidding format, the allocation and the pricing rules. 

%To exploit the available flexibility of players, we propose to use a combinatorial auction. In it, each player specifies for each of her possible consumption profiles, how much she wishes to get.


\subsection{Bidding format and allocation rule}
%\paragraph{Bidding Format}


In the proposed auction, each player expresses all her acceptable trading profiles and the utility associated with each one of them. To do so, each player bids a feasible set of consumption profiles $\hat{\fsi}$ (this can be done by bidding the battery capacity, initial state of charge and the player's demand $\hat{\li}$) and her utility function $\uti$, such as the one defined in Equation \eqref{eq:utility}. 
%and $ \hat{\guari} = \max_{\bai \in \hat{\fsi}} \uti(\bai + \hat{\li}, 0, 0)$. 
Here, we use the $\hat{h}$ notation to emphasize that the bid needs not to be truthful. From the bids, we can obtain $\displaystyle \hat{\guari} = \max_{\bai \in \hat{\fsi}} \uti(\bai + \hat{\li}, 0, 0)$, the maximum utility that player $i$ can guarantee without trading in the local market.

Observe that to bid the utility function $\uti$, it suffices to bid the set of buying and selling prices $\pbi, \pzi$.
%In the proposed auction, each players bids her feasible consumption set $\fsi$ and her cost function $\ci$ (buying and selling prices of the TEC). This bid, of course, need not to be truthful if the mechanism is not incentive compatible. 

%\subsection{Allocation rule}
%\paragraph{Allocation Rule}

%Before defining the allocation rule, observe that for any feasible consumption profile to be traded $f^i \in \fsi$, $\pri\oi = (\pri[1] \oi[1], \dots, \pri[t] \oi[t])$ defines a partial trade, i.e., at time-slot $t$, only $\pri[t] \oi[t]$ gets traded.
%Furthermore, observe that if a player trades only a proportion of his consumption profile, then the non-traded part will have to be paid at the TEC cost. Therefore, the cost incurred by trading partially is $\sum_{t=1}^T \ci[t]((1 - \pri[t])\oi[t]) + \gaini[t]$, where $P^i_t$ is the payment associated to the trading in the market.
%Finally, we will denote $\displaystyle \alpha^* = \min_{\oi \in \fsi} \uti(\oi)$, the largest utility that player $i$ can achieve with her reported cost and feasible set.
Regarding the allocation rule, it will be derived from the optimal solution of optimization problem $\eqref{eq:optallocrule}$.
%In this subsection we introduce the optimization problem \eqref{eq:optallocrule} from which the allocation rule is derived. 
%We have chosen this cost function as this is analogous 
We chose to maximize the value of the local trades as the objective of the allocation problem, as this is analogous to finding the clearing price in double auctions such as \cite{huang2002design}.

\begin{subequations}
\begin{alignat}{3}
\max_{\oi, \pri, \gaini }              &\quad&  \sum_{i \in N} \sum_{i \in \T} \ci[t]\left( \pri[t]\oi[t] \right)  &&          & \label{eq:optallocrule} \\
\text{subject to: } 
%&\quad& \gaini[t]   && \leq \ci[t](\pri[t]\oi[t]) &\quad \forall t \in \T \label{eq:allocruleoptcons1}\\
                    &\quad&  \sum_{i \in N} \gaini    && \geq 0    &\quad %\forall t \in \T 
                    \label{eq:allocruleoptcons2} \\
                    &\quad& \gaini + \sum_{t=1}^T \ci[t]\left[ 1 - \pri[t]\oi[t] \right]   &&  \leq -\hat{\resi}  &\quad \forall n \in \N \label{eq:allocruleoptcons3} \\
                    &\quad& \sum_{n \in \N} \pri[t]\oi[t] &&  = 0  &\quad \forall t \in \T \label{eq:allocruleoptcons4} \\
                    &\quad&  \oi   && \in \hat{\fsi} + \hat{\li}   &\quad \forall i \in \N \label{eq:allocruleoptcons5}
\end{alignat}
\end{subequations}


%The first constraint, \eqref{eq:allocruleoptcons1}, guarantees that all trades are as profitable in the market as with the TEC. 

The first constraint \eqref{eq:allocruleoptcons2} ensures that if the equality holds, all the money is redistributed among the participants according to the market decisions, while if the inequality is strict, the market maker obtains a profit. Constraint \eqref{eq:allocruleoptcons3} guarantees that the auction is individually rational, i.e., each players is at least as good as if she had not participated in the local market. It's important to note that encoding all of the $\N$ constraints \eqref{eq:allocruleoptcons3} requires a total of $\N\T$ additional binary variables. Equalities \eqref{eq:allocruleoptcons4} ensure that the amount of sold energy is equal to the energy bought in every time-slot. The last constraint guarantees that only feasible net consumption profiles are used. 
Finally, the amount of energy traded by player $i$ at time-slot $t$ is given by $\pri^*\oi^*$, where $\pri^*$ and $\oi^*$ are the optimal solutions of optimization problem \eqref{eq:optallocrule}. 

%Each of the $\N$ constraints requires, \eqref{eq:allocruleoptcons3},


\subsection{Payment rule}

As a payment rule, one alternative is to use the value of $\gaini^*$ in the optimal solution of \eqref{eq:optallocrule}. For the cases in which the values of $\gaini^*$ will not be unique, a predefined rule can be used to choose among the possible values. One such rule can be the set of values $\gaini^*$ that distributes the gains of the market as fairly as possible.
%Another alternative is to consider for each time-slot the maximum selling price and the minimum buying price among all players trading at time-slot $t$. Any value between that range is an acceptable trading price for the time-slot. 

We proceed to illustrate our proposal with an example.

%We proceed to introduce Example \ref{ex} which will be used again in Section 3 to illustrate the proposed auction.

\subsection{A simple example}
%\begin{example}\label{ex}
Consider two players $1$ and $2$ such that: $x^1 = (0, -1, 0)$, $x^2 = (0, 0, 1)$, $\beta^1 = \beta^2 = (2, 3, 3)$, $\zeta^1 = \zeta^2 = (1, 1, 1)$, $\mathbf{S}^1 = \mathbf{S}^2 = 1$, $\mathbf{S}^1_0 = \mathbf{S}^2_0 = 0$. If none of them trades in the market, their optimal utilities are given by $\alpha^1 = 1$ and $\alpha^2 = -2$, and their net consumption profiles by $n^1 = (0, -1, 0)$, $n^2=(1, 0, 1)$. 
%Furthermore, they are uniquely achieved  when their battery profile is exactly $s^1 = (1, 0, -1)$ and $s^2 = (0, 0.9, -0.9)$.

%Consider a player with load $\li = (-1, 0, 2)$, $\pbi = (2,2,2)$, $\pzi=(1,1,1)$ and $\storagei = 1$. Without trading in the market ($\pri = (0, 0, 0)$), the charging/discharging decision that maximizes the player utility is given by $\bai = (1, 0, -1)$, for a net load of $\oi = (0, 0, 1)$ and a total utility of $\guari = -2$.

%\begin{example}{Example 1 cont.}
We will now assume that the two players decide to participate in the auction and they do so truthfully. In the optimal solution of the allocation problem defined by their bids, it holds that $n^1 = (0, -1, 0) = - n^2$, $\lambda^1 = \lambda^2 = (0, 1, 0)$. Furthermore, the maximum value is attained at: $3\times (1)  + 1 \times (-1) = 2$

Regarding the payments, we have that for player 1: 
$\mathcal{P}^{1*} \leq -1$ and for player 2: $\mathcal{P}^{2*} \leq 2$.
Consequently, any payment from player $2$ to player $1$ in the interval $\mathcal{P}^{2*} \in (1, 2)$ will leave both players better off than before. 
%In this example, the fairest payment from $2$ to $1$ would be: $\mathcal{P}^{2*} = \frac{(1.9 + 4.4)}{2}$.
%We can imagine that we always use the middle point of the interval as the payment rule.

%Observe that the optimal profiles returned by the allocation problem would have never been chosen by each player alone, as they do not maximize their utility without payments.
%\end{example}


%\end{example}
\subsection{General properties of the solution}

First, observe that in \eqref{eq:optallocrule}, the scenario without trades ($\gaini[t] = \pri[t] = 0, \forall i, \ \forall t$) is always feasible and therefore, a solution exists. This solution needs not to be unique, as discussed in subsection 3.2.
Secondly, when all players bid truthfully, the proposed auction obtains the consumption and trading profiles that maximize the value of the trades. The obtained allocation outperforms the results obtained when players maximize their individually utility and attempt to trade later using sequential auctions. An example of this was given in the previous subsection. There, the total utility of players went from $-1$, had they tried to trade in sequential auctions using the net profiles that maximized their individual utilities, to $0$ by trading in the proposed auction.

%Solucion no unica
%Verdad da mejor

\section{Conclusion}

In this paper we introduced a combinatorial double auction to be used in local energy markets as a replacement to run several sequential auctions in the same day, one for each time-slot. The proposed model maximizes the value of the trades in the local market by exploiting the latent flexibility of the players, given that players bid truthfully.
%With a simple example, it was shown how the new model achieves to capture the latent flexibility of the players while increasing their utility. 
Future lines of research include variations of the proposed mechanism that are strategy-proof or that require less binary variables.

%Como vimos en la seccion 3.4, es un poco mejor. fairness..

% ---- Bibliography ----
%
% BibTeX users should specify bibliography style 'splncs04'.
% References will then be sorted and formatted in the correct style.
%
\bibliographystyle{splncs04}
\bibliography{mybib}
%

\end{document}
